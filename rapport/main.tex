\documentclass[12pt]{article} % text width
\usepackage[utf8]{inputenc} % encode text to utf8
\usepackage[T1]{fontenc} % use T1 font encoding for french

% paragraph formatting: https://www.overleaf.com/learn/latex/Paragraph_formatting
\setlength{\parindent}{1em}
\setlength{\parskip}{1em}

\usepackage[greek, french, english]
{babel} % text correction
\newcommand*{\textingreek}[1]{%
	\foreignlanguage{greek}{#1}%
} % new command for greek text
\newcommand*{\tig}[1]{\textingreek{#1}} % previous command shortcut
\usepackage[a4paper]{geometry}
\usepackage{graphicx}% for graphics

% quotes and bibliography: https://www.overleaf.com/learn/latex/Typesetting_quotations
\usepackage[
    left = \flqq{},% 
    right = \frqq{},% 
    leftsub = \flq{},% 
    rightsub = \frq{} %
]{dirtytalk}
\usepackage{csquotes}
\usepackage[
backend=biber,
style=numeric,
sorting=none
]{biblatex}
% add commands for automatic cite/uncite distinction
\DeclareBibliographyCategory{cited}
\AtEveryCitekey{\addtocategory{cited}{\thefield{entrykey}}}
\addbibresource{biblio.bib} % bibliography
\nocite{*} % all references

\title{PPH: Quelles sont les possibilités de faire de l'art avec du code ?}
\author{Florian Rascoussier}
\date{March 2022}

\pagenumbering{roman} % set page numbering of front matter sections

% use acronyms and glossaries
% toc: add glossary to table of contents
\usepackage{hyperref}
\usepackage[acronym, toc]{glossaries} 
\makeglossaries

\newacronym{asic}{ASIC}{Application-specific integrated circuit}
 % acronyms definitions

% document content
\begin{document}

\begin{titlepage}
    \begin{center}
        \vspace{2cm}

        \Large
        \textbf{PPH: Le code est-il de l'art ?}

        \vspace{2cm}
        \Large
        Problématique\\
        \textbf{Quelles sont les possibilités de faire de l'art avec du code ?}
            
        \vspace{2cm}
        \normalsize
        Florian Rascoussier

        \vfill

        \begin{figure*}[h]
            \centering
            \includegraphics[scale=0.7]{img/insa.pdf}
        \end{figure*}
        
    \end{center}
\end{titlepage}
\newpage

\begin{titlepage}
    \begin{center}
        \vspace{1cm}
        \Large
        \textbf{Projet Personnel en Humanité}

        \vspace{0.5cm}
        \normalsize
        \textbf{Institut National des Sciences Appliquées de Lyon (INSA Lyon)}\\

        \vspace{2cm}
        \Huge
        \textbf{Rapport de soutenance}

        \vspace{2cm}
        \Large
        Problématique\\
        \textbf{Quelles sont les possibilités de faire de l'art avec du code ?}


        \vspace{1cm}
        \normalsize
        Semestre d'été 2022
        
        \vspace{2cm}

        \vfill

        \Large
        \textbf{Autheur:}\\
        \vspace{0.5cm}
        \normalsize
        \textbf{Florian Rascoussier}\\
        4IF - INSA Lyon

        \vspace{1cm}
        \Large
        \textbf{Tuteur:}\\
        \vspace{0.5cm}
        \normalsize
        \textbf{Lionel Brunie}, Professeur et Directeur du Département d'Informatique de l'INSA de Lyon (IF), chercheur au laboratoire LIRIS\\

        \vspace{1cm}
        \Large
        \textbf{Jury:}\\
        \vspace{0.5cm}
        \normalsize
        \textbf{Lionel Brunie}
            
    \end{center}
\end{titlepage}
\newpage

\section*{Abstract}
Ce rapport de projet présente un ensemble de recherches et de réflexions personnelles pour tenter d'explorer la pratique de la programmation et le code qui en résulte sous l'angle de la création artistique. L'objectif est de comprendre ce qu'est le code et comment il peut être concu et créé de manière artistique. Il s'agit donc de saisir en quoi le code est-il de l'art.

En outre, le code étant un outil à même de former des images, des mouvements ou des sons, on ne s'intéressera pas à la manière de créer des oeuvres d'art en utilisant du code. Il s'agit de se focaliser sur le code et la programmation elle-même, et non pas sur le produit ou résultat de ces derniers.

\section*{Remerciements}
Remerciements particuliers aux encadrants de ce projet PhD Track, à savoir:

\begin{itemize}
    \item \textbf{Lionel Brunie} - <Lionel.Brunie@insa-lyon.fr>: Encadrant final  du PPH.
    \item \textbf{Gonzalo Suarez Lopez} - <gonzalo.suarez-lopez@insa-lyon.fr>: Premier encadrant du PPH.
\end{itemize}

\section*{Avant propos}
Ce rapport à été rédigé une première fois dans les semaines précédentes à une séance de débat. Celle-ci a permise de réunir diverses personnes: étudiant, professeurs et chercheurs en informatique ; afin de discuter de la problématique et prendre connaissances des recherches préalables avant de débattre sur le sujet et de proposer de nouvelles pistes de réflexion.

\newpage
\tableofcontents

\newpage
\pagenumbering{arabic} % reset page numbering

\section{Introduction}
Nous verrons d'abord comment la pratique de la programmation et le code qui en résultent ont évolué afin de comprendre en détail qu'est ce que le code. Nous nous intéresserons ensuite à l'art, et la relation entre le code et ce dernier. Enfin, nous exploreront les diverses méthodes, pratiques et techniques qui font du code un art à part entière.

\section{Qu'est-ce que le code ?}
\subsection{Programmation, code et langage}
La programmation est un terme générique. Composé du suffixe \textit{-ation}, du latin \textit{-atio}, utilisé pour signifier un action, et du nom \textit{programme} lui-même issue via le latin \textit{programma} du grec ancien \textgreek{pr'ogramma}.
qui peut désigner divers concepts selon le domaine, par exemple le terme \textit{programmation} désigne, dans le milieu du cinéma, l'action de déterminer les programmes ou films d'une salle donnée. Bien sûr, le terme est ici considéré sous l'angle des sciences informatiques. \say{Pour éviter cette confusion polysémique, des termes alternatifs sont parfois utilisés, comme le code ou le codage, mais la programmation informatique ne peut pas se réduire au code} \cite{Romero2017-mk}. D'après le Larousse, la programmation est donc: 
\say{
    \begin{quote}
        \begin{itemize}
            \item l'action de programmer, c'est-à-dire fournir à un ordinateur les données et les instructions conscernant un problème à résoudre, une tâche à effectuer, etc.
            \item l'établissement d'un programme, c'est-à-dire une séquence d'instructions et de données enregistrée sur un support et susceptible d'être traitée par un ordinateur.
        \end{itemize}
    \end{quote}
}
\cite{Nimmo2017-ya}. Cette définition n'est pas parfaite mais pose les éléments essentiels. On comprends que le code est l'object de la programmation, ce qu'elle produit.

Le code est donc un object issue d'un acte de production. En cela, la programmation relève donc de la technique tandis que le code est de l'ordre de la production, de la création. Le mot \textit{code} est cependant lui aussi fortement polysémique. En effet le terme est issue du latin \textit{codex}, variante de \textit{caudex} signifiant \say{tronc}. Le terme est cependant fortement associé aux livres et particulièrement aux ouvrages juridiques. Aujourd'hui, le terme peut prendre divers sens selon le domaine considéré, du droit à la linguistique en passant par l'informatique ou la cryptographie. Le Larousse distingue ainsi de nombreux sens différents pour le mot, dont les plus intéressantes du point de vue du sujet sont les suivantes:
\say{
    \begin{quote}
        \begin{itemize}
            \item Ensemble de règles qu'il convient de respecter.  
            \item Système de symbole permettant d'interpréter, de transmettre, un message, de représenter une information, des donnée.
            \item Système conventionnel, rigoureusement structuré, de symboles ou de signes et de règles combinatoires intégrées dans le processus de la communication.
            \item Ensemble d'instructions écrites dans un langage lisible par l'homme et devant être traduites en langage machine pour être exécuté par un ordinateur.
        \end{itemize}
    \end{quote}
}
\cite{Nimmo2017-ya}. Toutes ces définitions sont intéressantes, notamment en considérant le point de vue d'un non informaticien. On comprend que le code désigne donc un moyen de transmettre de l'information via un système de règles, mais aussi les règles elle-mêmes. 

On considèrera tout de même quelques définitions du Wikitionnaire:
\say{
    \begin{quote}
        \begin{itemize}
            \item Le code source est un texte qui représente les instructions de programme telles qu'elles ont été écrites par un programmeur.
            \item Le langage machine, ou code machine, est la suite de bits qui est interprétée par le processeur d'un ordinateur exécutant un programme informatique.
            \item Système de symboles permettant de représenter une information dans un domaine technique (code binaire,  alphanumérique, morse).
        \end{itemize}
    \end{quote}
}
\cite{noauthor_undated-vw}. Ces définitions viennent compléter la définition du code. Du point de vue de l'informatique, le code est donc un texte d'instructions et les instructions elles-mêmes indépendament du langage ou du niveau d'abstraction considéré. Ce code ne peut être écrit qu'en respectant des normes et des règles afin d'être finalement exécutable par une machine. Le code est le fruit d'un acte de création, d'un travail d'écriture et de conception.

Le code est ainsi fortement lié aux règles qui le définissent. On peut alors considérer que le code est un moyen de transmettre des informations, mais aussi de les représenter. Pour cela, il est nécessaire de respecter des règles, qui sont des normes qui définissent le langage de programmation. On introduira alors la notion de langage de programmation. Mais d'abord, afin de mieux comprendre la nature exacte du code et ses différentes formes, il convient de s'intéresser à son histoire.

\subsection{Petite Histoire du code}
La programmation est une technique dont les prémisses remontent au début du XIXème siècle, avec l'invention du métier à tisser Jacquard. Il faut cependant attendre le milieu du XXème siècle, et notamment les travaux du mathématicien anglais Alan Turing et son article fondateur de la science informatique, \citetitle{Turing1937-pn}, pour que cette technique se développe réellement.



\subsection{Le Langage}

Le terme \textit{langage} est souvent associé en premier lieu à une faculté intrinsèque de l'homme. En informatique, il désigne la façon dont instructions et données sont codées et de les manipuler. Le langage a enfin un sens différent du point de vue de l'artiste. Aborder le sujet du langage et du code requiert ainsi de bien comprendre les notions couvertes par le terme en question. 

Pour le dictionnaire Larousse, le langage à donc de multiples définitions:
\say{
    \begin{quote}
        \begin{itemize}
            \item Faculté propre à l'homme d'exprimer et de communiquer sa pensée au moyen d'un système de signe vocaux ou graphiques; et ce système.
            \item Système structuré de signes non verbaux remplissant une fonction de communication.
            \item Ensemble des procédés utilisés par un artiste dans l'expression de ses sentiments et de sa conception du monde.
            \item Mode d'expression propre à un sentiment, à une attitude.
            \item Ensemble de caractères, de symboles et de règles permettant de les assembler, utilisé pour donner des instructions à un ordinateur.
            \item (machine) Langague directement exécutable par l'unité centrale d'un ordinateur, dans lequel les instructions sont exprimées en code binaire.
        \end{itemize}
    \end{quote}
}
\cite{Nimmo2017-ya}. On remarque que ces définitions recoupent en parties celles que l'on a pu voir dans les parties précédentes au sujet du code. Code et langage sont ainsi très lié. Du point de vue du programmeur, le langage est en effet la langue qui définit comment organiser les instructions et les données afin de produire un programme \cite{BernardAmade2019}.

Un langage de programmation est un outil pour le programmeur. En effet, tout programme étant par définiton une suite d'instructions machines c'est-à-dire de 0 et de 1, il n'est pas aisé pour un humain d'écrire ses programmes directement dans ce langage bas niveau. Cependant, il est possible d'écrire des programmes dans des langages compréhensible par un humain mais qui puisse être tout de même être transformé en langage machine pour être exécuté par un ordinateur. De mêmes que les langues parlées, il existe une multitude de langages de programmation. Comme le dit \citeauthor{BernardAmade2019}, \say{Il existe des pratiques de programmation très différentes, il existe aussi des langages de programmation qui proposent des modes d'approche très différentes} \cite{BernardAmade2019}. La variété des premières expliquant naturellement la multiplicité des seconds, en plus d'autres facteurs comme les évolutions de la recherches, l'histoire jusque même aux goûts, habitudes et usages des programmeurs.

Le choix du ou des langages que l'on souhaite utiliser pour écrire un programme est donc un aspect important. Comme l'écrit \citeauthor{Dijkstra1976} dans son livre \citetitle{Dijkstra1976} en \citedate{Dijkstra1976}: \say{[...] one is immediately faced with the question: Which programming language am I going to use ?, and this is not a mere question of presentation! A most important, but also most elusive, aspect of any tool is its influence on the habits of those who train themselves in its use. If the tool is a programming language, this influence is, -whether we like it or not- an influence on our thinking habits.} \cite{Dijkstra1976}.

La création d'un langage de programmation est en elle-même un tâche complexe. En effet de très nombreux éléments rentrent en comptent dans la création d'un langage. Comment les rappellent les auteurs de \citetitle{GDowekJJLevy2006}, \say{Nous sommes encore très loin d'avoir trouvé un langage de programmation
définitif. Presque chaque jour, de nouveaux langages sont créés et de nouvelles fonctionnalités sont ajoutées aux langages anciens. [...] La première chose que l'on doit décrire, quand on définit un langage de
programmation, est sa syntaxe. Faut-il écrire x := 1 ou x = 1? Faut-il mettre des parenthèses après un if ou non ? Plus généralement, quelles sont les suites de
symboles qui forment un programme? On dispose pour cela d'un outil efficace : la notion de grammaire formelle. À l'aide d'une grammaire, on peut décrire la
syntaxe d'un langage de manière qui ne laisse pas de place à l'interprétation et qui rende possible l'écriture d'un programme qui reconnaît les programmes
syntaxiquement corrects. Mais savoir ce qu'est un programme syntaxiquement correct ne suffit pas
pour savoir ce qui se passe quand on exécute un tel programme. Quand on définit un langage de programmation, on doit don également décrire sa sémantique, c'est-à-dire ce qui se passe quand on exécute un programme. Deux langages peuvent avoir la même syntaxe mais des sémantiques différentes.} \cite{GDowekJJLevy2006}. Un langage de programmation est ainsi composé plusieurs éléments qui permettent à une machine de comprendre et d'excuter le programme sans ambiguité.

En plus ces considérations, les langages se définissent par la ou les grands approches qu'ils choisissent de considérer ou favoriser. Ce sont les paradigmes. Dans la conférence intitulée \citetitle{GerardBerry2015} en \citedate{GerardBerry2015} à l'INRIA de Rennes, \citeauthor{GerardBerry2015} reviens sur le processus créatif derrière la création des nombreux langages. Au départ, on peut considérer deux langages A et B différents avec des approches fondamentalent uniques l'un par rapport à l'autre. Chacun forme sa propre communauté d'utilisateur. Puis arrive un nouveau langage C qui ne propose pas un nouveau point de vue mais reprend les concepts des deux langages précédents pour tenter tant bien que mal de les associéer. Ce dernier viens grapiller des utilisateurs aux deux premiers. Les langages A et B originaux se mettent donc à incorporer des concepts de l'autre afin de récupérer des utilisateurs et venir proposer de nouvelles améliorations à ses utilisateurs.\say{C'est-à-dire que vous avez des tas de gens qui ont des idées complètement baroques. Vous avez des tas de groupes d'utilisateurs avec des traditions totalement différentes dans des pays qui n'ont rien à voir [...] et la paysage deviens très joyeusement incompréhensible. Et c'est là vie, parce que c'est un espace de création. Difficile.} \cite{GerardBerry2015}. \citeauthor{GerardBerry2015} propose ainsi une vision caricaturale mais néanmoins descriptive derrière l'apparition et l'évolution des langages informatiques.

Le langage, support essentiel de la programmation moderne, est donc en lui-même un \say{espace de création} \cite{GerardBerry2015} qui n'est pas sans influence sur le code en lui-même \cite{Dijkstra1976}. Au contraire, il représente un élément essentiel du code, de par la structure, la forme et finalement la philosophie qu'il impose nécessairement de considérer. Nous devrons donc explorer ce que cela implique sur le plan artistique. Cependant, il faut d'abord nous intéresser à la question de l'art lui-même.

\section{Le code est-il de l'art ?}

\subsection{L'art}
Bien difficile est le problème de la définition de l'art. Faut-il tenter d'en donner une définition précise, chercher à décrire ce qu'il n'est pas ou encore tenter d'apporter une certaine idée de la pensée qui l'inspire et le fait naître ? C'est que le terme peut facilement changer de sens d'une personne, d'un groupe, d'un pays ou d'une culture à l'autre.

\subsection{La programmation en tant que pratique}
\subsection{Le code comme medium}

\section{Faire de l'art avec du code ?}

\subsection{Code et poésie}
Le code se définissant le plus souvant comme étant un texte, il y a alors un parallèle naturel à faire entre texte litéraire ou poétique et texte de code informatique \cite{FCramer2001}.

\subsection{L'art des langages}
\subsection{Approches graphiques}
Typographie, estétique visuelle

\subsection{L'art de la modélisation}
\subsection{Artisanat et élégance}




\section{Conclusion}

\newpage
\section{Appendices}

\subsection{Obscurcissement du code}



% glossary and acronyms
\printglossary[type=\acronymtype]
\printglossary


\newpage
\printbibliography[
    heading=bibintoc,
    category=cited,
    title={Références}
]

% uncited references (bibliography)
% https://tex.stackexchange.com/questions/6967/how-to-split-bibliography-into-works-cited-and-works-not-cited
\printbibliography[
    notcategory=cited,
    heading=bibintoc,
    title={Bibliographie complémentaire},
]


\restoregeometry
\end{document}