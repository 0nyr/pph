\documentclass[12pt]{article} % text width
\usepackage[utf8]{inputenc} % encode text to utf8
\usepackage[T1]{fontenc} % use T1 font encoding for french

\usepackage[french]{babel} % text correction
\usepackage[a4paper]{geometry}

% quotes and bibliography
\usepackage{dirtytalk}
\usepackage{csquotes}
\usepackage[
backend=biber,
style=numeric,
sorting=ynt
]{biblatex}
\addbibresource{biblio.bib} % bibliography

\title{PPH: Quelles sont les possibilités de faire de l'art avec du code ?}
\author{Florian Rascoussier}
\date{March 2022}

\pagenumbering{roman} % set page numbering of front matter sections

% use acronyms and glossaries
% toc: add glossary to table of contents
\usepackage[colorlinks]{hyperref}
\usepackage[acronym, toc]{glossaries} 
\makeglossaries

\newacronym{asic}{ASIC}{Application-specific integrated circuit}
 % acronyms definitions

% document content
\begin{document}

\begin{titlepage}
    \begin{center}
        \vspace{2cm}

        \Large
        \textbf{PPH: Le code est-il de l'art ?}

        \vspace{2cm}
        \Large
        Problématique\\
        \textbf{Quelles sont les possibilités de faire de l'art avec du code ?}
            
        \vspace{2cm}
        \normalsize
        Florian Rascoussier

        \vfill

        \begin{figure*}[h]
            \centering
            \includegraphics[scale=0.7]{img/insa.pdf}
        \end{figure*}
        
    \end{center}
\end{titlepage}
\newpage

\begin{titlepage}
    \begin{center}
        \vspace{1cm}
        \Large
        \textbf{Projet Personnel en Humanité}

        \vspace{0.5cm}
        \normalsize
        \textbf{Institut National des Sciences Appliquées de Lyon (INSA Lyon)}\\

        \vspace{2cm}
        \Huge
        \textbf{Rapport de soutenance}

        \vspace{2cm}
        \Large
        Problématique\\
        \textbf{Quelles sont les possibilités de faire de l'art avec du code ?}


        \vspace{1cm}
        \normalsize
        Semestre d'été 2022
        
        \vspace{2cm}

        \vfill

        \Large
        \textbf{Autheur:}\\
        \vspace{0.5cm}
        \normalsize
        \textbf{Florian Rascoussier}\\
        4IF - INSA Lyon

        \vspace{1cm}
        \Large
        \textbf{Tuteur:}\\
        \vspace{0.5cm}
        \normalsize
        \textbf{Lionel Brunie}, Professeur et Directeur du Département d'Informatique de l'INSA de Lyon (IF), chercheur au laboratoire LIRIS\\

        \vspace{1cm}
        \Large
        \textbf{Jury:}\\
        \vspace{0.5cm}
        \normalsize
        \textbf{Lionel Brunie}
            
    \end{center}
\end{titlepage}
\newpage

\section*{Abstract}
Ce rapport de projet présente un ensemble de recherches et de réflexions personnelles pour tenter de répondre à la problématique...

\section*{Remerciements}
Remerciements particuliers aux encadrants de ce projet PhD Track, à savoir:
\begin{itemize}
    \item Lionel Brunie - <Lionel.Brunie@insa-lyon.fr>: Encadrant final  du PPH.
    \item Gonzalo Suarez Lopez - <gonzalo.suarez-lopez@insa-lyon.fr>: Premier encadrant du PPH.
\end{itemize}

\section*{Avant propos}
Ce rapport à été rédigé une première fois dans les semaines précédentes à une séance de débat. Celle-ci a permise de réunir diverses personnes: étudiant, professeurs et chercheurs en informatique ; afin de discuter de la problématique et prendre connaissances des recherches préalables avant de débattre sur le sujet et de proposer de nouvelles pistes de réflexion.

\newpage
\tableofcontents

\newpage
\pagenumbering{arabic} % reset page numbering

\section{Introduction}
La programmation est une technique dont les prémisses remontent au début du XIXème siècle, avec l'invention du métier à tisser Jacquard. Il faut cependant attendre le milieu du XXème siècle, et notamment les travaux du mathématicien anglais Alan Turing et son article fondateur de la science informatique, \citetitle{Turing1937-pn}, pour que cette technique se développe réellement.

Nous verrons d'abord comment la pratique de la programmation et le code qui en résultent ont évolué afin de comprendre en détail qu'est ce que le code. Nous nous intéresserons ensuite à l'art, et la relation entre le code et ce dernier. Enfin, nous exploreront les diverses méthodes, pratiques et techniques qui font du code un art à part entière.

\section{Programmation et code}

\subsection{L'Histoire du code}
\subsection{Qu'est-ce que le code ?}



\section{Le code est-il de l'art ?}

\subsection{L'art}
\subsection{La programmation en tant que pratique}
\subsection{Le code comme medium}

\section{Faire de l'art avec du code ?}

\subsection{L'art des langages}
\subsection{Approches graphiques}
\subsection{L'art de la modélisation}
\subsection{Artisanat et élégance}




\section{Conclusion}

\newpage
\section{Appendices}

\subsection{Obscurcissement du code}



% glossary and acronyms
\printglossary[type=\acronymtype]
\printglossary


\newpage
\printbibliography[
    heading=bibintoc,
    category=cited,
    title={Références}
]

% uncited references (bibliography)
% https://tex.stackexchange.com/questions/6967/how-to-split-bibliography-into-works-cited-and-works-not-cited
\printbibliography[
    notcategory=cited,
    heading=bibintoc,
    title={Bibliographie complémentaire},
]

\subsection{Bibliography} % to be removed


\restoregeometry
\end{document}