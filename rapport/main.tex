\documentclass[12pt]{article} % text width
\usepackage[utf8]{inputenc} % encode text to utf8
\usepackage[T1]{fontenc} % use T1 font encoding for french

% paragraph formatting: https://www.overleaf.com/learn/latex/Paragraph_formatting
\setlength{\parindent}{1em}
\setlength{\parskip}{1em}

\usepackage[greek, french, english]
{babel} % text correction
\newcommand*{\textingreek}[1]{%
	\foreignlanguage{greek}{#1}%
} % new command for greek text
\newcommand*{\tig}[1]{\textingreek{#1}} % previous command shortcut
\usepackage[a4paper]{geometry}
\usepackage{graphicx}% for graphics

% quotes and bibliography: https://www.overleaf.com/learn/latex/Typesetting_quotations
\usepackage[
    left = \flqq{},% 
    right = \frqq{},% 
    leftsub = \flq{},% 
    rightsub = \frq{} %
]{dirtytalk}
\usepackage{csquotes}
\usepackage[
backend=biber,
style=numeric,
sorting=none
]{biblatex}
% add commands for automatic cite/uncite distinction
\DeclareBibliographyCategory{cited}
\AtEveryCitekey{\addtocategory{cited}{\thefield{entrykey}}}
\addbibresource{biblio.bib} % bibliography
\nocite{*} % all references

\newcommand{\ts}{\textsuperscript} % superscript for 2nd or XIXème

\title{PPH: Quelles sont les possibilités de faire de l'art avec du code ?}
\author{Florian Rascoussier}
\date{March 2022}

\pagenumbering{roman} % set page numbering of front matter sections

% use acronyms and glossaries
% toc: add glossary to table of contents
\usepackage{hyperref}
\usepackage[acronym, toc]{glossaries} 
\makeglossaries
\newglossaryentry{calligramme}
{
    name=calligramme,
    description={Un calligramme est un poème ou type de poème dans lequel l'agencement du texte, des mots et des symboles forme une image en lien avec le contenu du poème, c'est-à-dire du sens du texte. Il s'agit donc d'un jeu sur le rapport entre forme et fond.}
}
\newglossaryentry{palindrome}
{
    name=palindrome,
    description={Un palindrome est un texte, une phrase, un nombre, un mot ou simplement une séquence de caractères qui peut se lire indifféremment dans un sens ou dans l'autre.}
}
\newglossaryentry{cypherpunk}
{
    name=cypherpunk,
    description={Ce terme lui-même désigne un individu et par extension une communauté d'individu qui milite pour l'utilisation de la cryptographie et de la protection forte de données et de l'anonymat sur internet. C'est également le titre d'un livre de Julian Assange : \textit{Cypherpunks: Freedom and the Future of the Internet}.}
}

\newacronym{asic}{ASIC}{Application-specific integrated circuit}
\newacronym{nft}{NFT}{Non Fungible Token}
\newacronym{ram}{RAM}{Random Access Memory}
\newacronym{ide}{IDE}{Integrated Development Environment}
\newacronym{ioccc}{IOCCC}{International Obfuscated C Code Contest}
\newacronym{foss}{FOSS}{Free and Open Source Software}
\newacronym{pgp}{PGP}{Pretty Good Privacy}
%\makeglossaries

\newacronym{asic}{ASIC}{Application-specific integrated circuit}
 % acronyms definitions, failed to make in work on a separate file!!!


% document content
\begin{document}

\begin{titlepage}
    \begin{center}
        \vspace{2cm}

        \Large
        \textbf{PPH: Le code est-il de l'art ?}

        \vspace{2cm}
        \Large
        Problématique\\
        \textbf{Quelles sont les possibilités de faire de l'art avec du code ?}
            
        \vspace{2cm}
        \normalsize
        Florian Rascoussier

        \vfill

        \begin{figure*}[h]
            \centering
            \includegraphics[scale=0.7]{img/insa.pdf}
        \end{figure*}
        
    \end{center}
\end{titlepage}
\newpage

\begin{titlepage}
    \begin{center}
        \vspace{1cm}
        \Large
        \textbf{Projet Personnel en Humanité}

        \vspace{0.5cm}
        \normalsize
        \textbf{Institut National des Sciences Appliquées de Lyon (INSA Lyon)}\\

        \vspace{2cm}
        \Huge
        \textbf{Rapport de soutenance}

        \vspace{2cm}
        \Large
        Problématique\\
        \textbf{Quelles sont les possibilités de faire de l'art avec du code ?}


        \vspace{1cm}
        \normalsize
        Semestre d'été 2022
        
        \vspace{2cm}

        \vfill

        \Large
        \textbf{Autheur:}\\
        \vspace{0.5cm}
        \normalsize
        \textbf{Florian Rascoussier}\\
        4IF - INSA Lyon

        \vspace{1cm}
        \Large
        \textbf{Tuteur:}\\
        \vspace{0.5cm}
        \normalsize
        \textbf{Lionel Brunie}, Professeur et Directeur du Département d'Informatique de l'INSA de Lyon (IF), chercheur au laboratoire LIRIS\\

        \vspace{1cm}
        \Large
        \textbf{Jury:}\\
        \vspace{0.5cm}
        \normalsize
        \textbf{Lionel Brunie}
            
    \end{center}
\end{titlepage}
\newpage

\section*{Abstract}
Ce rapport de projet présente un ensemble de recherches et de réflexions personnelles pour tenter d'explorer la pratique de la programmation et le code qui en résulte sous l'angle de la création artistique. L'objectif est de comprendre ce qu'est le code et comment il peut être conçu et créé de manière artistique. Il s'agit donc de saisir en quoi le code est-il de l'art.

En outre, le code étant un outil à même de former des images, des mouvements ou des sons, on ne s'intéressera pas à la manière de créer des œuvres d'art en utilisant du code. Il s'agit de se focaliser sur le code et la programmation elle-même, et non pas sur le produit ou résultat de ces derniers.

\section*{Remerciements}
Remerciements aux encadrants de ce projet, à savoir :

\begin{itemize}
    \item \textbf{Lionel Brunie} - <lionel.brunie@insa-lyon.fr> : Encadrant final  du PPH.
    \item \textbf{Gonzalo Suarez Lopez} - <gonzalo.suarez-lopez@insa-lyon.fr> : Premier encadrant du PPH.
\end{itemize}

Une première ébauche de ce rapport a d'abord été rédigé dans les semaines précédentes à une séance de débat. Celle-ci a permise de réunir quelques personnes du Département Informatique de l'INSA Lyon ; afin de discuter de la problématique et prendre connaissances des recherches préalables avant de débattre sur le sujet et de proposer de nouvelles pistes de réflexion.

Remerciements particuliers aux participants de la séance de débat, qui a eu lieu le jeudi 7 avril 2022 (10-12h) :

\begin{itemize}
    \item \textbf{Eric Guérin} - <eric.guerin@insa-lyon.fr> : Professeur, chercheur au laboratoire LIRIS.
    \item \textbf{Anicet Nougaret} - <anicet.nougaret@insa-lyon.fr> : Étudiant IF 4\ts{ème} année. 
    \item \textbf{Ithan Velarde Requena} - <ithan.velarde-requena@insa-lyon.fr> : Étudiant IF 3\ts{ème} année.
\end{itemize}


\section*{Avant-propos}
Ce projet de PPH est titré par la question \say{Le code est-il de l'art ?}. Ce titre décrit bien l'objectif et le cadre général de ce PPH mais couvre un vaste champ de réflexion qui n'est pas réellement représentatif du point de vue choisie. En effet, le code étant le produit d'une pratique, il semble assez direct de dire qui puisse être l'objet d'une création ou d'une pratique artistique. Ainsi, la problématique réelle est donc \say{Quelles sont les possibilités de faire de l'art avec du code ?}. Cette question est donc ciblée sur les pratiques, les formes et les moyens de création artistique ayant le code pour objet.

La réponse à la problématique ne pouvant passer que par la compréhension du sujet global, il va donc falloir consacrer les 2 premières parties du rapport à discuter de ce qu'est le code, et de la nature de l'art, avant de réellement pouvoir aborder les différents axes de création artistiques, et les pratiques connexes que l'on peut relier au code et à la programmation. En effet, art comme code sont des concepts flous qui cachent une multitude des réalités et concepts différents. Bien qu'il ne sera pas possible d'en donner une définition précise et complète, discuter de ces concepts est néanmoins nécessaire pour se doter des outils de compréhension et de réflexion utiles pour la suite du rapport.

Enfin, il semble important de rappeler que ce rapport est issue d'un projet de PPH, c'est-à-dire d'un travail personnel de 4\ts{ème} année, sans créneau horaire réservé dans la formation et valorisé au minimum d'un unique crédit ECTS. Il est donc important de considérer que ce rapport n'a en aucune manière l'ambition de faire un tour exhaustif de la question auquel il tente d'apporter un éclairage.

\newpage
\tableofcontents

\newpage
\pagenumbering{arabic} % reset page numbering

\section{Introduction}
Nous verrons d'abord tenteront de comprendre en quoi la pratique de la programmation et le code qui en résulte ont évolué afin de comprendre en détail ce qu'est le code. Nous nous intéresserons ensuite à l'art, et la relation entre le code et ce dernier. Enfin, nous explorerons les diverses méthodes, pratiques et techniques qui font du code un art à part entière.

\section{Qu'est-ce que le code ?}
Comprendre le code est la première étape indispensable afin de pouvoir considérer le problème du rapport entre celui-ci et l'art. Le terme n'est pas nécessairement bien connoté dans la sphère de l'informatique de même que l'on ne parle pas de rond, mais de cercle en géométrie cartésienne. Pourtant, le terme est intéressant en cela qu'il a le mérite d'être en apparence simple et donc à même de parler à chacun. Il faut cependant tenter d'éclaircir le terme et d'en apporter une définition satisfaisante.

\subsection{Programmation, code et langage}
La programmation est un terme générique. Composé du suffixe \textit{-ation}, du latin \textit{-atio}, utilisé pour signifier une action, et du nom \textit{programme} lui-même issue via le latin \textit{programma} du grec ancien \textgreek{pr'ogramma} qui peut désigner divers concepts selon le domaine, par exemple le terme \textit{programmation} désigne, dans le milieu du cinéma, l'action de déterminer les programmes ou films d'une salle donnée. Bien sûr, le terme est ici considéré sous l'angle des sciences informatiques. \say{Pour éviter cette confusion polysémique, des termes alternatifs sont parfois utilisés, comme le code ou le codage, mais la programmation informatique ne peut pas se réduire au code} \cite{Romero2017-mk}. D'après le Larousse, la programmation est donc : 
\say{
    \begin{quote}
        \begin{itemize}
            \item l'action de programmer, c'est-à-dire fournir à un ordinateur les données et les instructions concernant un problème à résoudre, une tâche à effectuer, etc.
            \item l'établissement d'un programme, c'est-à-dire une séquence d'instructions et de données enregistrée sur un support et susceptible d'être traitée par un ordinateur.
        \end{itemize}
    \end{quote}
}
\cite{Nimmo2017-ya}. Cette définition n'est pas parfaite mais pose les éléments essentiels. Elle montre que la programmation est une action, un acte qui suppose des considérations techniques (quelle machine, quel format, quelles instructions), physique ou structurelles (comment l'exécution est effectuée, comment fournir les instructions à la machine, quel support), et un but (quelle tâche, pour quoi faire). La programmation est donc une activité créatrice et créative. On comprend aussi que le code est l'objet de cette création, ce qu'elle produit.

Le code est donc un objet issu d'un acte de production. En cela, la programmation relève donc de la technique tandis que le code est de l'ordre de la production. Le mot \textit{code} est cependant lui aussi fortement polysémique. En effet, le terme est issu du latin \textit{codex}, variante de \textit{caudex} signifiant \say{tronc}. Le terme est cependant fortement associé aux livres et particulièrement aux ouvrages juridiques. Aujourd'hui, le terme peut prendre divers sens selon le domaine considéré, du droit à la linguistique en passant par l'informatique ou la cryptographie. Le Larousse distingue ainsi de nombreux sens différents pour le mot, dont les plus intéressantes du point de vue du sujet sont les suivantes :
\say{
    \begin{quote}
        \begin{itemize}
            \item Ensemble de règles qu'il convient de respecter.  
            \item Système de symbole permettant d'interpréter, de transmettre, un message, de représenter une information, des données.
            \item Système conventionnel, rigoureusement structuré, de symboles ou de signes et de règles combinatoires intégrées dans le processus de la communication.
            \item Ensemble d'instructions écrites dans un langage lisible par l'homme et devant être traduites en langage machine pour être exécuté par un ordinateur.
        \end{itemize}
    \end{quote}
}
\cite{Nimmo2017-ya}. Toutes ces définitions sont intéressantes, notamment en considérant le point de vue d'un non-informaticien. On comprend que le code désigne donc un moyen de transmettre de l'information via un système de règles, mais aussi les règles elles-mêmes. 

On considèrera tout de même quelques définitions du Wiktionnaire:
\say{
    \begin{quote}
        \begin{itemize}
            \item Le code source est un texte qui représente les instructions de programme telles qu'elles ont été écrites par un programmeur.
            \item Le langage machine ou code machine, est la suite de bits qui est interprétée par le processeur d'un ordinateur exécutant un programme informatique.
            \item Système de symboles permettant de représenter une information dans un domaine technique (code binaire,  alphanumérique, morse).
        \end{itemize}
    \end{quote}
}
\cite{noauthor_undated-vw}. Ces définitions viennent compléter la définition du code. Du point de vue de l'informatique, le code est donc un ensemble d'instructions et les instructions elles-mêmes qui sont alors dépendantes d'un langage, système et niveau d'abstraction considéré. En effet, tout code ne peut être écrit qu'en respectant des normes et des règles afin d'être finalement exécutable par une machine. Il est le fruit d'un acte de création, d'un travail d'écriture et de conception. Le code est donc fortement lié aux règles qui le définissent. 

Cet étallage de définitions permet donc de saisir la nature profondément polysémique du terme et son lien étroit avec la notion de programmation. On peut donc catégoriser les différents sens qui seront utiles dans la suite de ce rapport:

\begin{itemize}
    \item Séquence instructions et de données qui constituent un programme et qui ont vocation à être exécuté par un ordinateur. Il s'agit d'un sens général qui englobe une variété de représentation, du \textit{code source} aux \acrfull{asic}.
    \item Système de règles, de symboles et de conventions permettant de représenter une information. Il s'agit d'une considération générale de ce que peut désigner le terme. On pensera notamment aux langages informatiques et langages de programmation qui sont des éléments essentiels à la pratique de la programmation moderne et qui déterminent fortement la forme que prendra le code source.
    \item On désigne par \textit{code source}, l'ensemble des documents qui forment un programme avant toute forme de traitement. Ce code est alors réparti dans un ou plusieurs textes et documents écrits qui peuvent être rédigés et lus par un humain. Ils servent de base à divers procédés permettant de passer de ces documents à un ensemble de données et d'instructions exécutés par un ordinateur. 
    \item Le \textit{code machine} désigne les instructions, généralement après traitement, qui sont donc directement exécutées par le processeur d'un ordinateur. Elles n'ont pas vocations à être écrites ou lues par un humain et sont dites \textit{de bas niveau}. Elles sont également fortement dépendantes du jeu d'instruction, c'est-à-dire des instructions machines exécutables par le processeur considéré. Cet élément explique entre autre le besoin de disposer de systèmes de représentations de plus hauts niveaux, indépendants de la notion de jeu d'instruction. On notera que l'\textit{assembly}, c'est-à-dire du code machine écrit sous une forme lisible par un humain n'en est pas stricto sensu une forme alternative puisqu'il représente la même chose, mais nécessite quand même une conversion.
    \item Enfin, on introduira la notion de \textit{code-idée}, c'est-à-dire l'algorithme, l'objectif derrière le code source ou machine, l'idée indépendante de tout langage ou implémentation. Cette idée traverse les différents niveaux de code et nait en premier lieu dans l'esprit du programmeur lorsqu'il écrit ou lit un code source. Cette idée peut s'exprimer en une multitude de formes selon le choix du langage utilisé. D'une certaine manière, le choix du langage va cependant impacter la forme prise par l'idée et il y a donc une distinction entre l'idée, la forme et l'exécution effective.
\end{itemize}

Pour résumer, on peut considérer que le code est un moyen de transmettre des informations, mais aussi de les représenter. Ces informations constituent un programme, fruit du travail du programmeur et conçu avec un objectif particulier. Pour cela, il est nécessaire de respecter des règles généralement sous la forme d'un langage de programmation. Afin de mieux comprendre l'évolution de ces concepts et leurs différentes formes, il convient de s'intéresser à l'histoire.

\subsection{Petite Histoire du code}
Remonter dans l'histoire du code et de la programmation est important pour comprendre ses origines, ses différentes formes et ses évolutions. La programmation est une technique qui permet définir des règles et des comportements sans avoir à changer le système physique qui l'utilise. Selon la définition que l'on se donne et les exemples que l'on choisi de considérer, on peut remonter jusque dans l'antiquité avec les automates de Héron d'Alexandrie \cite{View_all_of_Hansels_posts2018-uw}. L'idée originale étant de pouvoir définir un système capable de modifier son comportement sans modifier ses composants physiques dans le but évident de multiplier les comportements sans avoir à changer le système.

La première invention majeure remonte cependant au début du XIX\ts{ème} siècle, avec l'invention du métier à tisser Jacquard, en 1801 à Lyon \cite{RDigest1982}. Son invention, inspirée des orgues de barbarie, permettait de lever automatiquement les fils de soie nécessaire à la réalisation de motifs. Motifs pouvant être modifiés grâce à un système modulaire de cartes perforées et d'un mécanisme en carré mobile \cite{noauthor_2009-bf}. Ce système est donc l'ancêtre direct des cartes perforées encore en usage au XX\ts{ème}siècle, et successivement amélioré par les équipes de Sir Thomas Watson et sa société IBM. Passant par étape de 22 à 80 colonnes et de 8 à 10 lignes, la fameuse \textit{IBM card} a permis à l'entreprise de se développer fortement dans la première moitié du XX\ts{ème} siècle, en restant pendant près de 40 ans le moyen par excellence pour stocker, transmettre et traiter des données \cite{noauthor_2012-xq}.

Entre temps, il est important de regarder les développements de Charles Baddage et ses travaux sur les machines calculatoires et analytiques qui font qu'il est souvent considéré, non sans raison, comme le père de l'ordinateur \cite{Copeland2020-my}. Inspiré par le métier Jacquard, ce professeur de l'université de Cambridge a d'abord créé le \textit{Difference Engine}, une machine entièrement analogique, composée de pièces mécaniques et capable de calculer automatiquement des tables de calculs pour certaines fonctions mathématiques comme le logarithme, ou encore pour le calcul automatique des marées. Bien qu'il n'acheva jamais sa machine, il proposa une nouvelle idée d'une machine généraliste qui aurait eu sa propre unité centrale de calcul et sa mémoire. 

Dans les notes accompagnant la traduction de notes d'un séminaire donné par Baddage en 1840, Augusta Ada King, comtesse de Lovelace, ou simplement Ada Lovelace proposa une méthode de calcul automatique de la suite des nombres de Bernoulli sur la machine analytique, sous forme d'un diagramme et de notes. Cette méthode décrite dans la note G, qui se distingue du pur calcul scientifique jusque-là envisagé, notamment par Baddage, et est souvent considéré comme le premier véritable programme informatique \cite{JKrysa}. Ada Lovelace est ainsi considéré comme la première programmeuse de l'histoire. 

Il faut cependant attendre le milieu du XX\ts{ème} siècle, et notamment les travaux du mathématicien anglais Alan Turing et son article fondateur de la science informatique, pour que cette technique se développe réellement \citetitle{Turing1937-pn}. La programmation allant de pair avec le développement et la montée en puissance des ordinateurs, on passe progressivement des programmes simples réalisés physiquement sur circuit électroniques, aux programmes sur cartes perforées, avant que ne se développent les premiers langages de programmations. Ces langages ont largement impacté la pratique de la programmation du fait du cadre et des outils qu'ils donnent aux programmeurs.

%TODO: histoire des langages de programmation

\subsection{Le Langage}

Le terme \textit{langage} est souvent associé en premier lieu à une faculté intrinsèque de l'homme. En informatique, il désigne la façon dont instructions et données sont codées et de les manipuler. Le langage a enfin un sens différent du point de vue de l'artiste. Aborder le sujet du langage et du code requiert ainsi de bien comprendre les notions couvertes par le terme en question. 

Pour le dictionnaire Larousse, le langage à donc de multiples définitions :
\say{
    \begin{quote}
        \begin{itemize}
            \item Faculté propre à l'homme d'exprimer et de communiquer sa pensée au moyen d'un système de signe vocal ou graphique ; et ce système.
            \item Système structuré de signes non verbaux remplissant une fonction de communication.
            \item Ensemble des procédés utilisés par un artiste dans l'expression de ses sentiments et de sa conception du monde.
            \item Mode d'expression propre à un sentiment, à une attitude.
            \item Ensemble de caractères, de symboles et de règles permettant de les assembler, utilisé pour donner des instructions à un ordinateur.
            \item (machine) Langage directement exécutable par l'unité centrale d'un ordinateur, dans lequel les instructions sont exprimées en code binaire.
        \end{itemize}
    \end{quote}
}
\cite{Nimmo2017-ya}. On remarque que ces définitions recoupent en parties celles que l'on a pu voir dans les parties précédentes au sujet du code. Code et langage sont ainsi très lié. Du point de vue du programmeur, le langage est en effet la langue qui définit comment organiser les instructions et les données afin de produire un programme \cite{BernardAmade2019}.

Un langage de programmation est un outil pour le programmeur. En effet, tout programme étant par définition une suite d'instructions machines, soit de 0 et de 1, il n'est pas aisé pour un humain d'écrire ses programmes directement dans ce langage bas niveau. Cependant, il est possible d'écrire des programmes dans des langages compréhensibles par un humain, mais qui puisse être tout de même être transformé en langage machine pour être exécuté par un ordinateur. De mêmes que les langues parlées, il existe une multitude de langages de programmation. Comme le dit \citeauthor{BernardAmade2019} : \say{Il existe des pratiques de programmation très différentes, il existe aussi des langages de programmation qui proposent des modes d'approche très différentes} \cite{BernardAmade2019}. La variété des premières expliquant naturellement la multiplicité des seconds, en plus d'autres facteurs comme les évolutions de la recherche, l'histoire et jusqu'aux goûts, habitudes et usages des programmeurs.

Le choix du ou des langages que l'on souhaite utiliser pour écrire un programme est donc un aspect important. Comme l'écrit \citeauthor{Dijkstra1976} dans son livre \citetitle{Dijkstra1976} en \citedate{Dijkstra1976}: \say{[...] one is immediately faced with the question: Which programming language am I going to use ?, and this is not a mere question of presentation! A most important, but also most elusive, aspect of any tool is its influence on the habits of those who train themselves in its use. If the tool is a programming language, this influence is, -whether we like it or not- an influence on our thinking habits.} \cite{Dijkstra1976}.

La création d'un langage de programmation est en elle-même une tâche complexe. En effet, de très nombreux éléments rentrent en comptent dans la création d'un langage. Comment les rappellent les auteurs de \citetitle{GDowekJJLevy2006} : \say{Nous sommes encore très loin d'avoir trouvé un langage de programmation définitif. Presque chaque jour, de nouveaux langages sont créés et de nouvelles fonctionnalités sont ajoutées aux langages anciens. [...] La première chose que l'on doit décrire, quand on définit un langage de programmation, est sa syntaxe. Faut-il écrire x := 1 ou x = 1 ? Faut-il mettre des parenthèses après un if ou non ? Plus généralement, quelles sont les suites de symboles qui forment un programme ? On dispose pour cela d'un outil efficace : la notion de grammaire formelle. À l'aide d'une grammaire, on peut décrire la syntaxe d'un langage de manière qui ne laisse pas de place à l'interprétation et qui rende possible l'écriture d'un programme qui reconnaît les programmes syntaxiquement corrects. Mais savoir ce qu'est un programme syntaxiquement correct ne suffit pas pour savoir ce qui se passe quand on exécute un tel programme. Quand on définit un langage de programmation, on doit don également décrire sa sémantique, c'est-à-dire ce qui se passe quand on exécute un programme. Deux langages peuvent avoir la même syntaxe mais des sémantiques différentes.} \cite{GDowekJJLevy2006}. Un langage de programmation est ainsi composé plusieurs éléments qui permettent à une machine de comprendre et d'exécuter le programme sans ambiguïté.

En plus ces considérations, les langages se définissent par la ou les grandes approches qu'ils choisissent de considérer ou favoriser. Ce sont les paradigmes. Dans la conférence intitulée \citetitle{GerardBerry2015} en \citedate{GerardBerry2015} à l'INRIA de Rennes, \citeauthor{GerardBerry2015} reviens sur le processus créatif derrière la création des nombreux langages. Au départ, on peut considérer deux langages A et B différents avec des approches fondamentalement uniques l'un par rapport à l'autre. Chacun forme sa propre communauté d'utilisateur. Puis arrive un nouveau langage C qui ne propose pas un nouveau point de vue, mais reprend les concepts des deux langages précédents pour tenter tant bien que mal de les associer. Ce dernier vient grappiller des utilisateurs aux deux premiers. Les langages A et B originaux se mettent donc à incorporer des concepts de l'autre afin de récupérer des utilisateurs et venir proposer de nouvelles améliorations à ses utilisateurs. \say{C'est-à-dire que vous avez des tas de gens qui ont des idées complètement baroques. Vous avez des tas de groupes d'utilisateurs avec des traditions totalement différentes dans des pays qui n'ont rien à voir [...] et le paysage deviens très joyeusement incompréhensible. Et c'est la vie, parce que c'est un espace de création. Difficile.} \cite{GerardBerry2015}. \citeauthor{GerardBerry2015} propose ainsi une vision caricaturale mais néanmoins descriptive derrière l'apparition, l'évolution et le foisonnement des langages informatiques.

Le langage, support essentiel de la programmation moderne, est donc en lui-même un \say{espace de création} \cite{GerardBerry2015} qui n'est pas sans influence sur le code en lui-même \cite{Dijkstra1976}. Au contraire, il représente un élément essentiel du code, de par la structure, la forme et finalement la philosophie qu'il impose nécessairement de considérer. Nous devrons donc explorer ce que cela implique sur le plan artistique. Cependant, il faut d'abord nous intéresser à la question de l'art lui-même.

\section{Le code est-il de l'art ?}
Une fois que la difficile question de la définition du code a été discutée, il reste encore la non-moins difficile question de la définition de l'art. La partie suivante tenter de monter toute la complexité de l'entreprise en essayant néanmoins d'établir un cadre utile pour les réflexions futures.

\subsection{Un problème de définition}
Bien difficile est le problème de la définition de l'art. Faut-il tenter d'en donner une version précise, chercher à décrire ce qu'il n'est pas ou encore tenter d'apporter une certaine idée de la pensée qui l'inspire et le fait naître ? C'est que le terme peut facilement changer de sens d'une personne, d'un groupe, d'un pays ou d'une culture à l'autre. Ainsi, il n'y a pas de définition précise de l'art \cite{SDavies1991}. Plutôt que chercher à en donner une définition, il vaut donc mieux se concentrer à sa compréhension et l'évolution de ses transformations. 

Pour \citeauthor{SDavies1991} dans \citetitle{SDavies1991}, l'art est une notion à la perception évolutive en fonction des époques et des auteurs: \say{In the past, art has been variously defined as imitation or representation(Plato 1955), as a medium for the transmission of feelings (Tolstoy 1995), as intuitive expression (Croce 1920) and as significant form (Bell 1914). Judged as essential definitions, these are unsatisfactory.} \cite{SDavies1991}. 

La définition de l'art et la distinction entre art et non-art a historiquement été contrôlée par les instances de pouvoir, politiques et religieuses mais aussi par les innovations des artistes eux-mêmes. Par exemple, l'artiste Marcel Duchamp, bien connu pour ses \textit{ready-mades} et notamment sa \textit{fontaine} qui n'est qu'un simple urinoir, se fait connaître en montrant que c'est l'idée et la démarche qui vont compter. En d'autres termes, ce qui va définir qu'un objet est artistique, n'est pas l'objet lui-même mais l'idée que celui-ci représente une forme d'art. Dès lors que l'artiste le décide, et que le spectateur le regarde comme tel, alors l'objet sera considéré comme artistique comme l'explique Arthur Danto dans \citetitle{ADanto1989} \cite{ADanto1989}. 

Malgré de ses évolutions, l'art reste un moyen d'expression, un langage à part entière avec des élaborations formelle et une force expressive signifiante. Il sert donc à communiquer comme illustré par le travail et les recherches de nombreux artistes. Par exemple, Christopher Pillault, artiste contemporain atteint d'autisme, qui se sert de l'art pour communiquer à défaut de pouvoir s'exprimer oralement. L'art peut permettre de transmettre des informations claires et ordonnées. L'art est un langage métaphorique, composé de connotations et de symboles, et qui s'appuie sur des analogies. 

La question de la définition de l'art pose en retour de nombreuses interrogations connexes. Parmi celles-ci se trouve celle de l'œuvre. Une œuvre se défini comme étant une chose matérielle ou immatérielle fabriquée par l'homme mais n'ayant aucune utilité fonctionnelle d'après le philosophe contemporain Heidegger, expert de la pensée rationnelle et de la mécanique philosophique. On peut citer l'exemple du tableau de René Magritte \citetitle{RMAgritte1929} qui cherche à représenter l'objet mais pas à l'utiliser, comme l'énonce la maxime \say{Ceci n'est pas une pipe} du tableau\cite{RMAgritte1929}. Le tableau en lui-même étonne, en jouant sur les règles implicites de la pensée et du rapport entretenu entre les choses, concepts et représentations. Pour le \citetitle{LarousseOnline-oeuvre}, une œuvre au sens relatif à l'art est une \say{production de l'esprit, du talent ; écrit, tableau, morceau de musique, etc., ou ensemble des productions d'un écrivain, d'un artiste : Les œuvres de Bach. Une œuvre d'art. Une thèse sur l'œuvre de Rimbaud.} \cite{LarousseOnline-oeuvre}. Cette définition est plus générale et ne se rattache pas directement à la notion d'art. Il s'agit plutôt d'une production, d'une création issue d'une pensée d'un auteur, d'un créateur ou d'un artiste.  

On le comprend, l'histoire de l'art est pleine de chamboulements et de transformations de la compréhension de ce qu'il est censé être. On peut alors penser que l'art ne peut alors être défini que par sa constance inhérente au changement et à la transformation. C'est le cas de Marcus Steinweg dans \citetitle{MSteinweg2009} qui explique que \say{art
asserts a consistency owed to its opening to inconsistency} (l'art affirme une consistance due à son ouverture à l'inconséquence). En abordant dans son essai que toute définition considérée, chacune est arbitraire puisqu'elle s'attache à ne prendre en compte qu'un ensemble restreint d'inconsistance dans une certaine perception de l'art, il montre que toute définition qui n'aurait pas pour sujet cet inconsistance même de l'art est nécessairement lacunaire donc arbitraire. Cette définition de l'art semble alors n'être pas réfutable en se rapprochant à sa manière du \textit{conatus} de René Descartes, de l'idée que toute chose est vouée au changement. Mais cette définition est-elle réellement satisfaisante en dépit d'être difficilement réfutable ?

Définir une relation entre art et œuvre est donc important mais là encore difficile.  L'art étant par essence un langage mystérieux, influencé à la fois par la pensée de l'artiste et celle du spectateur, celui-ci est donc source de questionnement. Il invite au dialogue intérieur, politique ou social de par la variété des interrogations, des ressentis et des réactions qu'il suscite. Du fait qu'il convoque les sentiments, l'art et la perception de l'art est avant tout personnelle. Sans tomber dans l'écueil du relativisme, il faut pourtant reconnaître que la définition de l'art est donc flexible et surtout dépendante de l'individu dont la subjectivité offre en elle-même une certaine légitimité dans la compréhension, l'appréciation et la création d'art. 

\subsection{Définir Art et Œuvre}
Définir l'art, de même que définir la relation entre art et œuvre est difficile. La partie précédente, bien qu'elle ait tentée à sa manière de présenter l'étendue des possibles et la complexité des réponses n'a pas permise de dégager un concensus. Afin d'éviter de tomber dans le relativisme, il va donc falloir néanmoins se donner des règles et des définitions un minimum satisfaisante. 

En reconnaissant l'impossibilité de définir consensuellement les concepts d'art et d'œuvre, il semble cependant correct de choisir un ensemble de définitions personnelles.

C'est ainsi que l'on définira pour la suite le concept d'\textit{art} de plusieurs manières. On peut d'abord s'inspirer des définitions courantes:
\begin{itemize}
    \item \say{Méthode pour faire un ouvrage, pour exécuter ou opérer quelque chose selon certaines règles.} \cite{WiktionnaireFr-art}
    \item \say{Reproduction par la main de l’homme ou la représentation de ce qui est dans la nature ; par opposition à naturel.} \cite{WiktionnaireFr-art}
    \item \say{Ensemble des procédés, des connaissances et des règles intéressant l'exercice d'une activité ou d'une action quelconque.} \cite{LarousseOnline-art}
    \item \say{Manière de faire qui manifeste du goût, un sens esthétique poussé} \cite{LarousseOnline-art}
    \item \say{Création d'objets ou de mises en scène spécifiques destinées à produire chez l'homme un état particulier de sensibilité, plus ou moins lié au plaisir esthétique} \cite{LarousseOnline-art}
\end{itemize}
Ces définitions sont intéressantes de par leur simplicité et leur rapport avec la réalité concrète. 

En plus de ces définitions de surface, on utilisera nos propres définitions. Avant cela, on se donnera l'expression \textit{auteur-créateur}, toute personne humaine physique qui utilise l'ensemble de ses connaissances et facultés dans un processus de création d'une chose. Ainsi, l'\textit{art} est donc:
\begin{itemize}
    \item Etat qui naît de la volonté d'un auteur-créateur du fait des caractéristiques qu'il donne à l'objet de sa création. Il naît d'une maîtrise technique, d'une volontée particulière et d'un parti pris.
    \item Méthodes, cadre utilisés par un auteur-créateur pour créer un objet ou une mise en scène.
    \item Elément considéré comme ayant des caractéristiques particulières qui font naître des sentiments chez un spectateur. Il suppose un dialogue entre œuvre et spectateur et des réactions chez ce dernier.
    \item Pratique qui nécessite de faire des choix qui aboutissent à l'établissement une création immatérielle ou réelle, éphémère ou pérenne dont la finalité peut être utile ou non mais qui peut être percue par un spectateur.
\end{itemize}
La notion d'art suppose en effet un dialogue. Dialogue d'abord entre un artiste et lui-même. Puis entre ce dernier et un public ou un spectateur imaginé. Et enfin entre l'œuvre elle-même et le spectateur. 

Ainsi, dans notre cadre, une \textit{œuvre} est donc simplement :
\begin{itemize}
    \item Une création considérée comme artistique, c'est-à-dire qui tendent à considéré que la création est de l'art pour le spectateur.
    \item Un objet qui peut être apprécié pour ses caractéristiques et qualités du fait d'un processus de création particulier.
\end{itemize}
On remarquera que les sens d'œuvre et d'art se recouvrent en partie puisqu'une œuvre peut être elle-même de l'art.

Il est important de rappeler qu'aucune de ces définitions n'est satisfaisante par elle-même. En outre, la somme de celles-ci ne l'est pas non plus. Elle est cependant utile dans le cadre précis de la compréhension du rapport entre programmation et code.

\subsection{Programmation et code} %Le code peut être de l'art... dire pourquoi...
Une fois que l'on s'est fixé un cadre suffisant pour comprendre les notions de programmation, code, art et œuvre, on peut alors dire simplement que \textbf{le code est à la programmation ce que l'œuvre est à l'art}. Cette phrase simple condense les parties précédentes en une maxime qui ne doit pas faire oublier les réserves et les problèmes mis au jour plus tôt.

On peut alors dire que \textbf{le code peut être de l'art dans la modalité que le code peut être œuvre et qu'œuvre peut être art}. Cette deuxième maxime est à comprendre dans les réflexions de la sous-partie précédente et notamment sur la relation sémantique entre les termes \textit{art} et \textit{œuvre}.

On peut comprendre ces deux maximes de différentes manières en fonction des définitions considérées: Le code en tant que système de règles ou en tant que langage est donc de l'ordre de la méthode. En ce sens, dire que \textit{le code est de l'art}, c'est dire qu'un ensemble de contraintes et de règles est en soi un art puisque art peut être cadre et méthode. C'est aussi impliquer qu'un système de l'ordre d'un langage de programmation, du fait des choix de conceptions et d'écriture qu'il impose est en lui-même de l'art, que la création de langages de programmation est aussi un art et que le cadre offert par le langage lui-même peut être considéré comme artistique. Il convient tout de même de rappeler les précautions qui s'impose : cela ne veut pas dire que tout langage de programmation est art. Il s'agit plutôt de réaliser que rien n'empêche et qu'il est possible de considérer qu'un langage est artistique.

La programmation étant une pratique créatrice qui impose des choix et nécessite souvent une prise de position et un point de vue particulier, il est naturel de la considérer comme une pratique artistique et donc de l'art. De même, le code en tant qu'objet, soit un ensemble d'informations par exemple sous forme de texte peut être une œuvre puisqu'il est issue d'une création artistique.

On remarque alors que la critique traditionnellement utilisée pour discréditer le code comme œuvre est en fait caduque. En effet, dire qu'une chose ne peut être de l'art au seul motif qu'elle peut avoir une finalité pratique n'a pas lieu d'être. D'une part, il existe du code qui n'existe sans aucune intention d'usage pratique ou de finalité utile réel. Tout code naît d'une idée et donc suppose un but mais ce but n'est nullement imposé d'être utile. D'autre part, le rapport entre art et utilité n'est qu'une conception de l'art dont les contre-exemples sont légion et qui n'entre même plus en considération dans les définitions modernes de l'art. En outre, l'art est souvent aux cœurs d'intérêts économiques importants du mécénat aux expositions en passant par de nouvelles variations comme les \acrshort{nft}.

Au regard de ces réflexions, il apparait que le code peut être de l'art. La notion de code doit alors s'entendre dans la complexité de ce qu'elle implique de même que la notion d'art. Un parallèle intéressant consiste à comparer le code et la musique.

% code et musique...
La musique se conçoit en effet sur différent support. Elle désigne l'art d'arranger un ensemble de sons entre eux dans le temps. Par extension, elle désigne aussi cet arrangement de son. C'est une pratique définie par un ensemble de règles physiques et conventions. La musique est donc un art de l'information. Ces informations peuvent être accédées directement par un humain en écoutant les sons lorsqu'ils sont produits par un chanteur, un instrument ou tout autre dispositif pouvant émettre des sons. Ces informations peuvent être stockées sur un support physique, par exemple les sillons d'un disque vinyle ou les micro-polarités d'un disque dur par exemple. Ces informations peuvent aussi être retranscrites sous forme de texte. Texte qui peut lui-même être stocké sous différentes formes ou être lu directement par un humain disposant des connaissances et capacités adéquates. 

Ainsi, la musique peut désigner à la fois l'art et l'œuvre sous différentes formes de représentations. On pourrait se simplifier la tâche en dissociant les concepts derrières 2 mots : la \textit{musique-pratique} et la \textit{musique-œuvre}. On peut alors les comparer à la programmation et au code. En effet, de même que la musique-pratique, la programmation dispose nécessairement de règles et contraintes. Puisqu'il s'agit de programmer des machines électriques, la programmation se voit contrainte par les principes de la physique. L'immense majorité des ordinateurs étant basés sur des calculs physiques en binaire, la programmation se doit donc de considérer d'emblée les contraintes physiques des machines en question. C'est ainsi que le programmeur doit souvent faire face à ces contraintes, par exemple en termes de vitesse d'écriture de disque, de temps de réaction ou de calcul, ou encore de l'espace mémoire disponible à différent niveau (cache, \acrshort{ram}, disque ou cloud...). La musique, elle doit faire avec les perceptions humaines et l'acoustique et la physique des ondes sonores. 

Afin d'encadrer la pratique de la musique, de nombreux cadres et règles ont étés développés. En informatique et donc en programmation, on trouve aussi de nombreux concepts et paradigmes qui offrent un cadre et un ensemble d'outils d'analyse et de construction aux programmeurs. L'objectif final est exclusivement la production d'un code dans le but d'être exécuté. Code qui se compose d'instruction à plus ou moins haut niveau en fonction des niveaux d'abstractions qui séparent le travail du programmeur de celui de la machine physique finale qui exécute les instructions. La programmation est donc aussi art de l'information. C'est aussi un certain art de la traduction. Il faut passer d'une idée humaine, exprimée en langage oral complexe et qui est souvent très imprécise et contextuelle, à un langage très simple, basés sur des règles mathématique et physiques. 

De même que le musicien doit s'entraîner pendant des années afin de parvenir à la maîtrise de son instrument afin de transmettre au mieux ses idées et sentiments par la musique, le programmeur doit lui aussi pratiquer pendant des années et surmonter les difficultés intellectuelles et techniques afin de mener à bien ses projets. Et de même en musique qu'un musicien particulièrement talentueux pourra très bien surpasser de nombreux autres musiciens, un programmeur chevronné est tout à fait à même de surpasser une équipe complète. Ceci ne doit pas faire oublier que bien souvent, que ce soit dans la sphère de l'Open Source ou dans le cadre professionnel ou même amical, la programmation se pratique rarement seule mais plutôt en équipe. Et de même que le musicien peut faire de la musique son métier, la programmation peut tout aussi bien relever du gagne-pain que du passe-temps. 

Enfin, la programmation produit un code, une fois que l'idée initiale à réussie à être exprimée d'une manière ou d'une autre en code qui peut être aisément converti en code-machine directement exécutable. Le code intermédiaire, exprimé d'une façon lisible pour un humain est souvent réalisé dans un langage de programmation mais aurait vraisemblablement une forme différente dans un autre langage de programmation. De nombreux langages ayant des propriétés similaires, le choix d'un langage plutôt qu'un autre laisse une certaine marge à la subjectivité et finalement constitue un premier choix important dans la pratique de la programmation. S'ensuit de nombreux autres choix tout au long du processus de développement, encadré par des principes, des goûts personnels, des paradigmes et des pratiques considérés comme plus ou moins bonnes. Le résultat final est donc le produit d'un processus complexe et créatif qui dispose de toutes les caractéristiques d'une œuvre. Si tous les codes ne sont pas nécessairement artistiques, un beau code, un bel algorithme ou un beau programme est parfaitement capable de susciter des réactions sur un spectateur averti ou non. De par l'aspect complexe et souvent cryptique qu'il semble avoir, fais l'objet d'une certaine crainte mêlée de fascination sur le grand public, ce dont se sont emparées de nombreuses œuvres à l'exemple des films de science-fiction tels que The Matrix ou TRON.

\section{Faire de l'art avec du code ?}
Cette partie s'éloigne d'un travail de compilation et de discussion purement journalistique ou philosophique. L'art touchant à la dimension personnelle et profonde de l'individu, la section suivante prend clairement la partie d'être plus subjective parce qu'elle tente de parler d'art au sens personnel. Il est de plus difficile d'étayer les arguments par des sources pertinentes. Cette section tentera de présenter les considérations et les moyens de l'art du code. 

\subsection{La programmation en tant qu'art} % la démarche, maitrise technique vs choix personnels et esthétiques, artisanat
Maintenant qu'on a pu établir pourquoi le code peut être de l'art, on va pouvoir discuter du comment. Il ne s'agit donc pas de s'intéresser littéralement à \textit{faire de l'art avec du code}, mais plutôt de \textit{faire du code avec de l'art}. En effet, on a montré précédemment que le rapport entre programmation et code se rapportait à celui qui existe entre art et œuvre et c'est donc plutôt dans le sens de la seconde proposition qu'il faut l'entendre. Dans cette partie, on s'intéressera d'abord au fond, c'est-à-dire à ce que fait, ou est censé faire le code. 

La programmation à donc pour principal objet l'écriture et la conception de code. Le rapport entre programmation et code est lui-même sujet à débat en fonction de l'interlocuteur et de la situation. Selon que celui-ci considère plus ou moins la pratique ou la théorie, il n'aura alors pas la même approche de la pratique. En effet, si on se place plus d'un point de vue théorie, par exemple en considérant une approche formelle de l'algorithmique et de la programmation, alors les règles que l'on va tendre à considérer sont d'ordre logique. On pourra par exemple se placer dans un certain domaine théorique, par exemple le lambda-calcul. Les considérations pratiques telles que le choix de tel ou tel langage devient alors secondaire. À l'autre bout du spectre, un programmeur de systèmes embarqué se doit de connaître au mieux les caractéristiques techniques du système en question afin de pouvoir le programmer au mieux. Dans ce cas, le choix des outils tels que le langage, ou les processus et outils de tests et de débogages deviennent des considérations de premier plan. Le résultat primera alors face au respect strict des règles théoriques. 

La nature du projet et de l'idée initiale est bien souvent déterminante dans l'approche retenue. En outre, des considérations éloignées peuvent néanmoins amener des solutions et des approches similaires puisqu'il n'existe pas réellement de meilleure façon de faire, et que celles considérées comme potentiellement bonnes peuvent se recouper. L'ensemble des choix guidant le développement seront responsables du résultat final, du code-source produit. Il n'existe malheureusement aucune garantie de succès et un projet informatique, qu'il soit personnel ou le fruit d'années de labeurs pour des centaines de personnes peut se solder par un échec et un abandon pur et simple de celui-ci. Dès lors, lorsque le code fonctionne bien, il n'est pas difficile d'éprouver des sentiments positifs et artistiques. La programmation est donc l'art de maîtrise la complexité des possibles et des contraintes afin de produire un code qui répond à des objectifs et un point de vue particulier.

% Maîtriser la complexité modélisation, langages, concepts, paradigmes, Propreté, élégance et simplicité, efficacité, maintenabilité, artisanat, R C Martin
À la manière de l'écrivain qui commence immanquablement son périple par la feuille blanche et qui peut perdre courage face à l'immensité des possibles, le programmeur moderne commence par un fichier vide. Au départ, il y a souvent une idée, un concept ou un objectif. Dans un cadre plus professionnel, le travail commence généralement bien en amont. Il s'agit d'abord de formaliser les attentes des clients, les besoins, le cas d'utilisations. Puisqu'il s'agit de dialoguer in-fine avec la machine, rien ne doit être laissé au hasard si bien que l'étape de conception initiale peut représenter à elle seule la majeure partie du travail. Le code n'arrivant qu'en fin de cycle une fois que tous les concepts ont été formalisés. 

Malgré tout, toutes ces étapes viennent simplifier la programmation, ou tout du moins lui fournir les bases nécessaires pour la faciliter. C'est que le programmeur doit jongler avec de nombreuses règles et concepts : modélisation, langages et outils, paradigmes, élégance, simplicité, efficacité, maintenabilité entre autres. Ces critères permettent de comprendre ce qu'est le \textit{clean-code}, c'est-à-dire le beau code. Le code beau aura pour effet de rendre la vie du programmeur et de ses successeurs plus agréable à défaut d'être plus facile. À l'inverse, un code sera dit sale s'il ne respecte pas ces règles de beauté, ce qui aura souvent des conséquences négatives pour l'avenir du projet. Il y a donc souvent un intérêt réel à programmer élégamment. Il pourrait alors s'agir de critères purement à but fonctionnel mais ce n'est pas le cas. En effet, ces règles de beautés reposent sur des partis pris, des réflexions et des goûts qui, s'ils peuvent reposer sur des réflexions réelles, n'en demeurent pas moins sujettes à interprétation et à évaluation. Chacun ayant plus ou moins ses propres règles et considérations de premiers plans. 

Le code et l'algorithme, c'est-à-dire le \textit{code-idée} et le \textit{code-source} peuvent chacun être une oeuvre artistique. Si l'algorithme est particulièrement ingénieux, ou à l'inverse inutilement complexe à l'extrême par exemple dans le cadre d'une compétition informatique alors il peut faire naître un sentiment de surprise, d'étonnement ou d'émerveillement pour devenir une oeuvre. Cependant bien souvent, ce code se retrouvera souvent noyé dans le reste du code-source. Il devient alors difficile de distinguer et de découvrir l'oeuvre noyée dans le reste. Car comprendre la beauté d'un code n'est pas facile et demande généralement un effort significatif de compréhension. Dans le cas d'une personne qui lirait le code source rapidement sans réellement chercher à comprendre les détails de ce qu'il se passe réellement, cherche l'oeuvre dans le reste du code s'apparente à cherche un poème au milieu d'un livre de recettes de cuisine. Dans ce cas la poésie viendrait d'une recette, originale et surprenante pour un cuisinier. Cela montre malheureusement qu'il faudra lire et comprendre chaque recette afin d'en mesurer pleinement la portée artistique ou non.s

Le code, et notamment la volonté du programmeur et le contexte dans lequel il s'inscrit peut donc lui donner toutes les caractéristiques d'une œuvre d'art. Cela n'est cependant pas automatique. Il reste à admettre que la frontière entre art et non-art peut-être difficile à déceler. Tant que l'expérience reste personnelle cela ne pose pas de problème, mais la confrontation de ses opinions avec celles d'autres personnes peut permettre de mettre au jour ou non cette frontière. Il existe cependant des moyens plus directs de créer du code artistique.

\subsection{Approche esthétique, forme et fond} % graphisme, couleur, forme, typographie, estétique visuelle
Passer du fond à la forme est un moyen simple et efficace de distinguer un code banal d'une œuvre artistique. Pour le programmeur, il y a souvent une vision qui prime sur le reste et une bonne manière envisagée pour faire les choses. Selon que le programmeur soit d'emblé dans une volonté artistique, cette vision entraîne une démarche qui peut se relever cruciale dans le projet.

Le processus de programmation, de par sa complexité, impose de faire des choix qui bien que guidé par une certaine idée du bien faire, nécessite des choix arbitraires et personnels parfois difficilement justifiables autrement que par esthétisme. Il peut en aller ainsi pour le choix d'un langage plutôt qu'un autre, d'un framework, d'une solution ou d'un \acrshort{ide}. On pensera aux solutions JetBrains par exemple et leur aspect poussé du design qui est clairement un argument de vente. Lorsqu'on est programmeur avec des objectifs d'efficacité en tête, il faut faire attention de ne pas trop se concentrer sur ces considérations sous peine de risque de se détourner du but premier et d'utiliser beaucoup plus de ressources que nécessaire sur des aspects non directement liés aux objectifs initiaux. Il peut être ainsi tentant de tout faire soi-même, selon ses propres idéaux de beauté et de forme. C'est ainsi que Donald Knuth, afin d'écrire ses livres tel que \citetitle{DKnuth1997} \cite{DKnuth1997} en est venu à développer de nombreux éléments afin de le faire selon ses propres esthétiques. Knuth à donc développé TeX, sur lequel est basé \LaTeX avec lequel ce rapport a été rédigé. Il a également créé de quoi définir sa propre police : \textbf{Computer Modern} qui est la police utilisée dans ses livres et également celle de ce document (police par défaut de \LaTeX). 

Mais Knuth est loin d'être le seul exemple et on pensera à l'exemple du jeu vidéo \textit{The Witness} dont le processus de développement est en lui-même une prouesse artistique plus que technique. Le jeu sera développé pendant 7 ans par l'équipe de Jonathan Blow, qui est à la fois le concepteur, directeur, designeur du projet, et surtout programmeur principal du projet. Celui-ci refuse d'utiliser un moteur de jeu disponible sur le marché et se lance avec son équipe dans le développement de son propre moteur de jeu 3D. En effet, il souhaite avoir le contrôle total de son jeu. De même, quand des problématiques de gestion des versions du code apparaissent, il décide de développer son propre système de contrôle de version. Les zones du jeu et les éléments de ce dernier sont entièrement sérialisés en texte afin de faciliter la gestion des conflits d'édition. Après le succès du jeu, Blow s'est lancé dans le développement de son propre langage de programmation et d'un moteur dédié. Il travaillerait actuellement sur 2 jeux utilisant ses propres technologies. \cite{Takahashi2018-cb}.

Les exemples précédents sont des illustrations de l'importance que la forme peut avoir dans le processus de développement dans lequel la programmation occupe une place de premier plan. L'esthétique peut recouvrir de nombreux éléments : langage de programmation, framework, librairie, \acrshort{ide}, moteur de jeu, typographie, choix des couleurs, etc. En outre, la programmation moderne impliquant le plus souvent la production d'un texte, on peut donc lui appliquer les mêmes techniques d'esthétique utilisées pour produire un texte par exemple en poésie. Il y a alors un parallèle naturel à faire entre texte littéraire ou poétique et texte de code informatique \cite{FCramer2001}.

On pensera entre autre au \gls{calligramme} qui propose un rapport intéressant sur le rapport entre la forme et le fond. En programmation, il existe notamment un exemple célèbre de code en C dont le texte prends la forme d'un cercle et qui une fois compilé et exécuté, affiche un tore en 3D grâce à du texte. Les exemples similaires sont nombreux et bon nombres d'exemples proviennent de l'\acrshort{ioccc}, une compétition internationale de programmation C dont l'objectif est de produire le code le plus surprenant et difficile à comprendre possible. De nombreux sites permettent de se rendre compte de l'ingéniosité et de la beauté réelles de ces codes. Dans les codes proposés, on trouvera entre autre :
\begin{itemize}
    \item Une implémentation du moteur analytique de Charles Babbage qui calcule les factorielles, les PGCD et les facteurs d'un nombre ; par Sean Barrett \cite{SBarrett-Babble}
    \item Un programme qui converti l'entrée standard en code morse. Le code lui-même est composé essentiellement de DAH DIT DAH qui épelle un message en code morse ; par Jim Hague \cite{JHague-morse}
    \item Un traducteur anglais-cochon-latin. Le code source a lui-même la forme d'une tête de cochon. Par Don Dodson \cite{DDodson-piglatin}
    \item Un programme dont chaque ligne de code est un \gls{palindrome}. (auteur inconnu). 
\end{itemize}
La liste continue. 

Ces programmes se rapprochent fortement de la poésie, par la forme textuelle et le rapport entre la forme et le fond, et encore au delà, à travers leur comportement une fois le code exécuté. Avec ces exemples, on peut mieux comprendre la dimension artistique que peut prendre le code. Dès lors, on comprend que le code peut être le produit d'approches esthétiques artistiques variées. Au départ réservé à un groupe d'initiés, ces nouvelles formes d'art se sont progressivement distillées dans la société notamment par internet. Le code est ainsi passé progressivement d'un mal nécessaire à une forme artistique à part entière qui regroupe ses propres courants et approches.

\subsection{Univers visuels, connotations et perceptions}
On l'a dit, l'essor de l'informatique et notamment la démocratisation des ordinateurs personnels, puis le développement d'Internet ont dès leur origine baignés dans la culture des sphères d'initiés à ces nouvelles technologies. On pensera par exemple au terme de \textit{geek} et \textit{nerd} qui se sont répandus. D'abord connotés négativement, ces termes ont peu a peu été utilisés par différentes communautés qui se sont alors appropriés ces termes. Très tôt, ces communautés ont développé leurs propres esthétiques et leur influence n'a fait que se renforcer avec l'essor des réseaux sociaux de masses. Au milieu de tous ces changements, une constante : le développement de la programmation et l'apparition de nouveaux outils plus simples et accessible, permettant au plus grand nombre de se former sur ces sujets.

C'est ainsi que le code est progressivement entré dans la culture populaire. De par son aspect rigide et sa complexité, le code fait peur aux yeux du grand publique. Dès lors, les personnes capables de le maîtriser apparaissent alors tour à tour comme des génies solitaires ou des criminels. La contre-culture des années 1980-1990 infuse alors l'idée du délinquant encapuchonné qui terrorise entreprises et gouvernement au nom de l'anarchisme et d'un rejet de la société civile. Le modèle du hacker est né. C'est dans cette optique qu'apparaissent des films comme \textit{The Matrix} ou \textit{Elysium} qui font du hacker le véritable héros car seul initié capable de lutter contre le système. Ces films ayant connu un certain succès, cette esthétique a profondément influencé la décennie suivante, particulièrement dans le milieu du jeu vidéo. La science-fiction s'est très tôt emparée de ces problématiques avec bons nombres d'œuvres de \textit{TRON} à \textit{Terminator}, en passant par 

L'esthétique du code s'est profondément infusée dans la culture web et les mêmes. On retrouve plusieurs esthétiques qui se croisent ou parfois s'affrontent. Puisque le code est souvent produit par une minorité, on retrouve une première connotation autour de l'espionnage, de la surveillance de masse et des complots. On se souviendra entre autre de la légende urbaine du jeu Polybius, un jeu vidéo censé causer des effets pouvant aller jusqu'à la mort, à l'IA maléfique GLaDOS dans les jeux Portal, ou au fameux Dark Net relié aux \glsplural{cypherpunk}, espions et criminels passant par des réseaux parallèles comme Tor. De l'autre côté du spectre, on retrouvera des esthétiques plus ancrées dans la pop-culture, aux univers colorés et mélodiques. La connotation sera alors beaucoup plus positive, centrée sur l'aspect communautaire. On peut noter la prévalence du jeu vidéo, véritable catalyser des univers développés autour ou incorporant des éléments liés au code et à la programmation.

Enfin, il semble important de mentionner l'impacte de la communauté Open Source et \acrshort{foss}. Avec des figures importantes comme Linus Torvalds, programmeur bien connu pour être le créateur et programmeur principal du noyau Linux et du système de contrôle de version Git. Cette communauté prône des valeurs d'accessibilité, de contrôle raisonné des données, de décentralisation et de liberté de l'information. Cette communauté fait une utilisation importante du code y compris à des fins de liberté d'expression. Ainsi, puisque le code est un texte, celui-ci peut être imprimé et tombe sous les lois de la liberté d'expression et de la presse. Il existe plusieurs exemples frappant d'utilisation du code à des fins de contournement qui pourrait être qualifiés d'artistique et qui trace un chemin direct entre code et littérature. Ainsi, Florian Cramer, dans son essai \citetitle{FCramer2001} propose l'exemple du code source du programme \acrshort{pgp}. Ce programme ayant été considéré comme une arme au sens légale, son auteur, Phil Zimmerman à alors eu l'idée d'en publier le code-source dans un livre. Ainsi, le programme est alors tombé dans le domaine littéraire et protégé par le \textit{U.S. First Amendment of free speech}: \say{So the book could be
exported outside the United States and, by scanning and retyping,
translated back into an executable program} \cite{FCramer2001}.

Ces communautés ont développé au fil des ans leurs propres univers et grammaires visuelles dans lequel le code prend généralement une place non négligeable. Ces esthétiques ont en retour influencé les programmeurs eux-mêmes. C'est ainsi que les outils de développement intègrent de plus en plus des notions d'esthétisme visuel, avec par exemple le thème sombre très affectionné par les développeurs pour être considéré comme plus agréable lors d'un travail généralement nocturne. De même, il est normal qu'une personne passant ses journées entières cherche à esthétiser son environnement de travail soit par pur plaisir soit par réel goût pour l'esthétisme. La police d'écriture peut par exemple être créée et utilisée exclusivement pour certains types de code particuliers ou par certains développeurs. Ces considérations se retrouvent historiquement dans la communauté des développeurs web du fait de leur intérêt nécessairement plus élevé pour les interfaces graphiques étant donné que c'est généralement une composante importante de leur métier. 

On remarque donc l'importance en filigrane que le code infuse dans la culture populaire et internet moderne. Porté par diverses communautés et univers visuels, influençant les œuvres notamment audio-visuelles ou littéraires. Programmation et code sont devenues des composantes importantes de la culture actuelle.  

\section{Conclusion}
À la question : \textit{Le code est-il de l'art ?}, la réponse est évidente. Oui, le code, en tant que produit d'une technique créative peut être de l'art, ce qui ne veut pas dire que tout code est nécessairement de l'art. Discuter du sujet impose d'avoir conscience de la complexité des notions mises en jeu. D'une part le code désigne plusieurs réalités et est à mettre en relation avec la programmation. D'autre part la définition de l'art pose de nombreux problèmes. Malgré tout, le code reste un espace de création et de dialogue à part entier, et l'art peut s'insinuer à différents niveaux : de complexité, de forme ou de fond, ou même graphique et visuelle.

Code, programmation, et informatique sont des disciplines ou des objets relativement nouveaux au vu de leur histoire encore très récente, surtout en comparaison avec d'autres disciplines artistiques anciennes. Malgré tout, ces derniers ont vu un développement fulgurant et ont eu un impact absolument majeur sur l'humanité et la vie courante. En ouvrant un nouvel espace de création artistique, code et programmation sont devenus des sujets de discussion, de débat, et de création artistique dont l'impact est déjà mesurable dans la culture actuelle. Il reste cependant de très nombreuses possibilités de développement et d'ouverture pour que le code devienne réellement un art et une forme d'expression reconnue, notamment du point de vue institutionnel et académique. À l'heure actuelle, si le code peut être de l'art, il est encore difficile de le considérer intrinsèquement comme tel.

\newpage
\section{Appendices}

% glossary and acronyms
\printglossary[type=\acronymtype]

\printglossary


\newpage
\printbibliography[
    heading=bibintoc,
    category=cited,
    title={Références}
]

% uncited references (bibliography)
% https://tex.stackexchange.com/questions/6967/how-to-split-bibliography-into-works-cited-and-works-not-cited
\printbibliography[
    notcategory=cited,
    heading=bibintoc,
    title={Bibliographie complémentaire},
]


\restoregeometry
\end{document}